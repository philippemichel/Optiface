\newabbreviation{ph}{ph}{praticien hospitalier}
\newabbreviation{ch}{ch}{centre hospitalier}
\newabbreviation{naco}{naco}{Nouveaux anticoagulants oraux}
\newabbreviation{adl}{adl}{Score ADL ( Activities of Daily Living) de \textsc{Katz}}
%
\newabbreviation{anova}{\textsc{anova}}{analysis of variance}
\newabbreviation{aic}{aic}{Akaike information criterion}
\newabbreviation{icc}{icc}{Coefficient de Corrélation Intraclasse}




\newglossaryentry{alpha}{name={Risque $\alpha$}, description={Probabilité de rejeter à tort l'hypothèse nulle càd conclure à une différence alors qu'il n'y en a pas.}}

\newglossaryentry{puissance}{name={Puissance}, description={1-$\beta$, $\beta$ étant la probabilité de rejeter l'hypothèse nulle quand elle est fausse càd conclure à l'absence de différence alors qu'elle existe.}}


\newglossaryentry{analyse post-hoc}{name={analyse post-hoc}, description={Un test post hoc ou test à postériori d’analyse de variance est un test utilisé après un test global (ANOVA par ex.) significatif. En effet, une ANOVA indique qu’il existe au moins un groupe qui diffère significativement des autres mais ne spécifie pas le quel. Les tests post hoc sont conçus pour comparer les groupes entre eux et indiquer quel groupe diffère de l’autre}}

\newglossaryentry{kruskal}{name={\textsc{Kruskal-Wallis}}, description={Le test de Kruskal-Wallis , aussi appelé ANOVA unidirectionnelle sur rangs, est une méthode non paramétrique (pouvant donc être utilisé sur de petits échantillon sans faire d'hypothèse de normalité) utilisée pour tester s'il existe une différence entre deux ou plusieurs groupes indépendants.}}

 \newglossaryentry{tukey}{name={\textsc{Tukey}}, description={Le test de Tukey permet d'effectuer une comparaison multiple en une seule étape en prenant en compte le risque d'erreur lié aux comparaisons multiples . Il est  souvent utilisé après une régression ,(ANOVA) ou autre test avec plus de deux niveaux dans le cadre d'un test post-hoc.}}

 \newglossaryentry{fkappa}{name={\textsc{Fleiss-Kappa}}, description={Le score de Fleiss-Kappa mesure l’accord entre plus de deux observateurs lors d'un codage qualitatif en catégories. On retient souvent une limite à 0,6 pour un accord \emph{fort}}}
